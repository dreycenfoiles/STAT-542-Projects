\documentclass{article}
\usepackage{graphicx}
\usepackage{physics}
\usepackage{float}
\usepackage{amsfonts}
\title{Project 3}
\author{Dreycen Foiles [dfoiles2] \\ Olek Yardas [yardsol2]}
\date{\today}
\begin{document}
\maketitle

\section{Introduction}

For this project, we are investigating how we can predict the sentiment of a movie review automatically by only using a relatively small vocabulary of words. This is an interesting project because it is our first attempt at natural language processing which is an incredibly important aspect of current machine learning focus. To accomplish this task, we use some of the techniques recommended by Professor Feng. We were surprised to find that highly accurate sentiment prediction models can be created without using advanced techniques such as neural networks. We were able to get results that surpassed the benchmarks using only logistic regression. 

\section{Methods}

The project consisted of two main distinct parts. The first part where we extracted data from the \verb,alldata.tsv, file. The second part where we used the extracted data to train a model to predict the sentiment of a movie review. In the following two sections, we will discuss the methods we used to accomplish these tasks.

\subsection{Vocabulary Generation}

To generate our vocabulary, we use the \verb,text2vec, R package to tokenize the words that are found in the \verb,alldata.tsv, file. To better separate the different words, we define a series of stop words. We can then further prune our vocabulary by removing all words from our vocab list that do not appear at least 10 times. Once our vocabulary list is pruned, we can turn it into a document term matrix (DTM). A DTM is a matrix where each row represents a document (in this case, a movie review) and each column represents a word. The value in each cell is the number of times that word appears in that document. We can then use this DTM to train our model.

We then use the t-test method to trim our vocuabulary to 2K. We use this test to
determine if the mean of the positive sentiment reviews in the document term
matrix have the same mean as the mean of the negative sentiment reviews in the
document term matrix. We choose the 2K words that have the highest t-statistic. We then use this trimmed vocabulary to create a new vectorizer and subsequently a new DTM. After constructing a new DTM, use \verb,GLMNET, to fit a logistic regression between the sentiment of a review and words in the DTM. After the fitting, if a word has a $\beta$ of less than 1000, we add it to our final vocabulary list. At the end of the vocab generation process, our vocabulary is 935 words long. 

\subsection{Sentiment Prediction}

To do the final sentiment prediction, we must first preprocess the reviews like how we did for the vocabulary generation. This involves separating by stop words, vectorizing and prune words that do not appear often enough. Once we have the pruned DTM, we can run it through LASSO regression using the model that we trained on the vocabulary generation. We then use the coefficients of the model to predict the sentiment of the review. We then compare the predicted sentiment to the actual sentiment and calculate the accuracy of the model.

\section{Results}

All runs were performed on a custom desktop PC with an Intel i5-9400 2.9 GHz and 16 GB of RAM. 

\subsection{Vocabulary Generation}

Final vocabulary size: 935. Time to generate vocab 4 minutes 13 seconds. Some examples of words in our final vocab list include: 

`actors', `adds', `adds\_to',`10\_10', `surprisingly\_good', `this\_bad',`waste',
`wrong', and `would\_recommend'


\subsection{Sentiment AUC on Splits}

\begin{center}
    \begin{tabular}{|c|c|c|}
        \hline
        Fold & Runtime (s.) & AUC \\ 
        \hline \hline 
        1 & 15.995 & 0.964 \\ 
        \hline
        2 & 15.392 & 0.964 \\ 
        \hline
        3 & 18.329 & 0.963 \\ 
        \hline
        4 & 15.387 & 0.964 \\ 
        \hline
        5 & 15.422 & 0.963 \\ 
        \hline
    \end{tabular}
\end{center}

\section{Conclusion}

In this project, we used rudimentary natural language processing techniques to create a model that can predict the sentiment of a movie review. Using powerful preprocessing and simple LASSO regression, we are able to accurately predict the sentiment of a review with only a relatively small vocabulary. This is an important result beacuse it shows that complicated models like neural networks are not always necessary to achieve high accuracy on comlicated tasks like natural language processing. 


\end{document}